\documentclass[10pt]{article}
% Include statements
\usepackage{graphicx}
\usepackage{amsfonts,amssymb,amsmath,amsthm}
\usepackage[sort]{natbib}
\usepackage[margin=1in,nohead]{geometry}
\usepackage{multirow,rotating,array}
\usepackage{algorithm,algorithmic}
\usepackage{pdfsync}
\usepackage{hyperref}
\hypersetup{backref,colorlinks=true,citecolor=blue,linkcolor=blue,urlcolor=blue}
\renewcommand{\qedsymbol}{$\blacksquare$}
\setlength{\parindent}{0cm}
\setlength{\parskip}{10pt}


% For sequential numbering
\newcounter{lecnum}
\renewcommand{\thepage}{\thelecnum\ -\ \arabic{page}}
\renewcommand{\thesection}{\thelecnum.\arabic{section}}
\renewcommand{\theequation}{\thelecnum.\arabic{equation}}
\renewcommand{\thefigure}{\thelecnum.\arabic{figure}}
\renewcommand{\thetable}{\thelecnum.\arabic{table}}



% Theorem environments (\autoref compatible)
\usepackage{aliascnt}
\newtheorem{theorem}{Theorem}[lecnum]

\newaliascnt{result}{theorem}
\newtheorem{result}[theorem]{Result}
\aliascntresetthe{result}
\providecommand*{\resultautorefname}{Result}
\newaliascnt{lemma}{theorem}
\newtheorem{lemma}[lemma]{Lemma}
\aliascntresetthe{lemma}
\providecommand*{\lemmaautorefname}{Lemma}
\newaliascnt{prop}{theorem}
\newtheorem{proposition}[prop]{Proposition}
\aliascntresetthe{prop}
\providecommand*{\propautorefname}{Proposition}
\newaliascnt{cor}{theorem}
\newtheorem{corollary}[cor]{Corollary}
\aliascntresetthe{cor}
\providecommand*{\corautorefname}{Corollary}
\newaliascnt{conj}{theorem}
\newtheorem{conjecture}[conj]{Conjecture}
\aliascntresetthe{conj}
\providecommand*{\conjautorefname}{Corollary}
\newaliascnt{def}{theorem}
\newtheorem{definition}[def]{Definition}
\aliascntresetthe{def}
\providecommand*{\defautorefname}{Definition}
\newaliascnt{ex}{theorem}
\newtheorem{example}[ex]{Example}
\aliascntresetthe{ex}
\providecommand*{\exautorefname}{Example}


\newtheorem{assumption}{Assumption}
\renewcommand{\theassumption}{\Alph{assumption}}
\providecommand*{\assumptionautorefname}{Assumption}

\def\algorithmautorefname{Algorithm}
\renewcommand*{\figureautorefname}{Figure}%
\renewcommand*{\tableautorefname}{Table}%
\renewcommand*{\partautorefname}{Part}%
\renewcommand*{\chapterautorefname}{Chapter}%
\renewcommand*{\sectionautorefname}{Section}%
\renewcommand*{\subsectionautorefname}{Section}%
\renewcommand*{\subsubsectionautorefname}{Section}% 


% My Macros
\def\indep{\perp\!\!\!\perp}
\newcommand{\given}{\mbox{ }\vert\mbox{ }}
\newcommand{\F}{\mathcal{F}}
\newcommand{\Expect}[1]{\mathbb{E}\!\left[#1\right]}
\renewcommand{\P}{\mathbb{P}}
\newcommand{\R}{\mathbb{R}}
\newcommand{\X}{\mathcal{X}}
\newcommand{\B}{\mathcal{B}}
\DeclareMathOperator*{\Variance}{Var}
\newcommand{\Var}[1]{\Variance\!\left[#1\right]}
\DeclareMathOperator*{\Covariance}{Cov}
\newcommand{\Cov}[1]{\Covariance\!\left[#1\right]}
\newcommand{\Y}{\mathcal{Y}}
\newcommand{\U}{\mathcal{U}}
\newcommand{\Radn}{\hat{\mathfrak R}_n}
\newcommand{\Rad}{\mathfrak R_n}
\newcommand{\norm}[1]{\left\lVert #1 \right\rVert}
\newcommand{\email}[1]{\href{mailto:#1}{#1}}
\DeclareMathOperator*{\argmin}{argmin}
\DeclareMathOperator*{\argmax}{argmax}
\newcommand{\indicator}{\mathbbm{1}}
\newcommand{\cdist}{\rightsquigarrow}
\newcommand{\cprob}{\xrightarrow{P}}
\newcommand{\clp}{\xrightarrow{L_p}}
\newcommand{\cas}{\xrightarrow{as}}
\renewcommand{\bar}{\overline}
\renewcommand{\hat}{\widehat}
\renewcommand{\tilde}{\widetilde}


% Your new macros



% To be entered
\setcounter{lecnum}{16}
\newcommand{\lecturer}{Prof.\ McDonald}
\newcommand{\scribe}{Yisu Peng}
\newcommand{\chtitle}{Empirical Risk Minimization (ERM algorithm)}
\newcommand{\lecdate}{17 October 2017}


\begin{document}
\rule{6.5in}{1pt}

\textsc{STAT--S 782
        \hfill \thelecnum\ --- \chtitle
        \hfill \lecdate}

\textsc{Lecturer: \lecturer \hfill Scribe: \scribe}
\rule{6.5in}{1pt}

\section{Empirical risk}
First, we introduce some notations,
\begin{itemize}
\item Data point $(x, y) \sim P \in \mathcal{P}$, \\
where $P$ is the joint distribution and $\mathcal{P}$ is the space of distributions.
\item Loss function $\ell : \mathcal Y \times \mathcal U \rightarrow [0, 1]$,\\
$\Y$ is the space of true labels and $\U$ is the space of predicted labels
\item Hypothesis space $\mathcal{F} : \mathcal{X} \rightarrow \mathcal{U}$
\item Data $Z^n = (Z_1, ... , Z_N) \overset{iid}\sim P$
\end{itemize}

\begin{definition}
some function $f \in \mathcal{F}$, the empirical risk is
\begin{align}
\hat{R}_n(f) = \frac{1}{n} \sum_{i=1}^{n} \Expect{\ell(y_i, f(x_i))}
\end{align}
\end{definition}

Now we have,
\begin{align*}
\Expect{\hat{R}_n(f)} &= \frac{1}{n} \sum_{i=1}{n} \Expect{\ell (y_i, f(x_i))}\\
	&= \Expect{\ell(y_i, f(x_i))}\\
	&= R(f)
\end{align*}

So based on the Hoeffding inequality,
\begin{align*}
&P(|\hat{R}_n(f) - R(f)| > \epsilon) \le 2 e^{-2n\epsilon^2}
\end{align*}

$\implies$ w.p. $1-2e^{-2n\epsilon^2}$
\begin{align*}
&|\hat{R}_n(f) - R(f)| < \epsilon
\end{align*}

We could find $\hat R_n (f)$ for any $f \in \mathcal{F}$, then why not $\hat f_n = \argmin_{f\in \mathcal F} \hat R_n (f)$

Here is how the logic goes: $f^* := \argmin_{f\in \mathcal F} R(f)$, and\\
\begin{align*}
R(\hat f_n) \approx \hat R_n(\hat f_n) \approx
\hat R_n(f^*) \approx R(f^*)
\end{align*}
However, thid doesn't work. We need conditions on $\mathcal{P}, \mathcal{F}, \ell$. First we intoduce the "induced loss class",
\begin{align*}
\mathcal{L}(\mathcal{F})=\{\ell(\cdot, f(\cdot)) : f \in \mathcal{F}\}
\end{align*}
and
\begin{align*}
q(n, \epsilon) := \sup_{P \in \mathcal{P}} P^n ( Z^n \in \mathbb{Z}^n: \sup_{f\in\mathcal{F}}|\hat R_n(f) - R(f)| \ge \epsilon)
\end{align*}
This is corresponding to the worst case probability distribution over all $\mathcal{P}$ of "bad" samples (those with worst/large risk diviation over $\mathcal{F}$).

We say $\mathcal{L}(\mathcal{F})$ has "uniform convergence of empirical means" (UCEM) property w.r.t. $\mathcal{P}$, if
\begin{align*}
\lim_{n\rightarrow\infty} q(n,\epsilon) = 0 \qquad \forall \epsilon > 0
\end{align*}

\begin{theorem}
if $\mathcal{L}(\mathcal{F})$ has UCEM property the ERM is PAC.
\end{theorem}

\begin{proof}
\begin{align*}
R(\hat f_n) - R(f^*) &= R(\hat f_n) - \hat R_n(\hat f_n) + \hat R_n(\hat f_n) - \hat R_n(f^*) + \hat R_n(f^*) - R(f^*)\\
\hat R_n(\hat f_n) - \hat R_n(f^*) &\le 0 \quad \text{by ERM}\\
R(\hat f_n) - \hat R_n(\hat f_n) &\le \sup_f ( R(f) - R_n(f)) \le \sup_f | R(f) - \hat R_n(f) |\\
\hat R_n(f^*) - R(f^*) &\le \sup_f ( R(f) - R_n(f)) \le \sup_f | R(f) - \hat R_n(f) |
\end{align*}
So, 
\begin{align*}
R(\hat f) - R(f^*) \le 2 \sup_f | R(f) - \hat R_n (f) |
\end{align*}
then with the UCEM property we have, $\exists n_0 (\epsilon, \delta)$ s.t.
\begin{align*}
q(n, \epsilon / 2) &\le \delta \qquad \forall n > n_0
\end{align*}
\end{proof}

\section{ERM algorithm}

Now, let's consider an algorithm,
\begin{align*}
r_{\mathcal{A}}(n,\epsilon) &= \sup_{P \in \mathcal{P}} P^n (Z^n: R(\hat f) \ge R(f^*) + \epsilon)\\
&\le \sup_{P \in \mathcal{P}} P^n(Z^n : \sup_{f\in\mathcal{F}} | \hat R_n(\hat f) - R(f) | \ge \epsilon / 2)\\
&= q(n, \epsilon / 2) \le \delta
\end{align*}

UCEM is sufficient for the ERM algorithm to ``work''. Then our question is, when do we have UCEM? It turns out $\Expect{\sup_{f\in\mathcal{F}} | R(f) - \hat R_n(f) |} = o(1)$ is sufficient for UCEM when $\ell : \Y \times \U \rightarrow [0, M]$.

Further more, we define
\begin{align*}
\Delta_n = \sup_f | R(f) - \hat R_n(f) |
\end{align*}
Then we have the following with McDiarmid's inequality,
\begin{align*}
P^n(\Delta_n - E \Delta_n \ge t) \le e^{-2nt^2} \qquad \forall \ell \rightarrow [0,1]
\end{align*}

What we want is $P^n(\Delta_n \ge \epsilon) \rightarrow 0$
\begin{align*}
P^n(\Delta_n \ge \epsilon) &= P^n(\Delta_n - E \Delta_n \ge \epsilon - E \Delta_n)\\
\exists n &\text{s.t.} E \Delta_n < \epsilon/2\\
	&\le P^n(\Delta_n - E \Delta_n \ge \epsilon / 2) \le e^{-n\epsilon^2/2}\\
	&\le f \qquad \text{with n large enough}
\end{align*}
When is $E\Delta_n$ is small?

\section{Rademacher complexity}
\begin{theorem}
ERM satisfies
\begin{itemize}
\item $R(\hat f_n) \le R(f^*) + 2 \Delta_n(\F)$,
\item $R(\hat f_n) \le \hat R_n(\hat f_n) + \Delta_n(\F)$.
\item $\Delta_n = \sup_{f\in\mathcal{F}} | R(f) - \hat R_n(f) |$
\end{itemize}
\end{theorem}

We need to control $\Delta_n$,

\begin{definition}
\begin{align}
\Radn (\F) = \mathbb{E}_\sigma\left[\left. \sup_{f\in\mathcal{F}} | \frac{2}{n} \sum_{i=1}{n} \sigma_i f(Z_i)| \right| Z^n\right]
\end{align}
$\sigma_i$'s are Rademacher variables, $P(\sigma_i = 1) = P(\sigma_i = -1) = \frac{1}{2}$
\end{definition}

\begin{itemize}
\item Think of this as ``projection" of $\F$ on to $Z^n$
\item ``Correlation" of $\F$ with noise
\item $\sup_f|f| \le M$ or this doesn't work
\item ``empirical Rademacher complexity"
\end{itemize}

\begin{definition}
Expected Rademacher complexity
\begin{align*}
\Rad(\F) = \mathbb{E}_{P^n} \left[ \Radn (\F) \right]
\end{align*}
\end{definition}

\begin{theorem}
\begin{align*}
\Expect{\Delta_n(\F)} \le \Rad (\F)
\end{align*}
\end{theorem}

\begin{proof}
(by symmetrization) First we need $Z^n$ and ``ghost" samples $\tilde{Z}^n$
\begin{align*}
\Delta_n(\F) &= \sup_{f\in\mathcal{F}} | R(f) - \hat R_n(f) |\\
	&= \sup_{f\in\mathcal{F}} \tilde{\mathbb{E}} \left[ | \tilde{\hat R}_n(f) - \hat R_n (f) | \right]\\
	&\le \tilde E \left[ \sup_{f\in\mathcal{F}} | \tilde{\hat R}_n (f) - \hat R_n (f) | \right]
\end{align*}
So,
\begin{align*}
\mathbb{E}_{Z^n} \Delta_n(\F) &= \mathbb{E}_{Z^n} \mathbb{E}_{\tilde Z^n} \left[ \sup_{f\in\mathcal{F}} | \tilde{\hat R}_n(f) - \hat R_n(f) | \right]
\end{align*}

Because $Z^n$ and $\tilde Z^n$ are independent, with the symmetrization
\begin{align*}
\tilde{\hat R}_n(f) - \hat R_n(f) &= \frac{1}{n} \sum_{i=1}^{n} f(\tilde Z_i) -f(Z_i)\\
&\overset{dist}{=} \frac{1}{n} \sum_{i=1}^{n} f(Z_i) - f(\tilde Z_i)\\
&\overset{dist}{=} \frac{1}{n} \sum_{i=1}^{n} \sigma_i ( f(Z_i) - f(\tilde Z_i) )
\end{align*}

So,
\begin{align*}
\mathbb{E}_{Z^n} \Delta_n(\F) &= \mathbb{E}_{Z^n} \mathbb{E}_{\tilde Z^n} \left[ |\frac{1}{n} \sum_{i=1}^{n} \sigma_i ( f(Z_i) - f(\tilde Z_i) )| \right]\\
 &\le \mathbb{E}_{Z^n\sigma} \left[ \sup_{f\in\mathcal{F}} |\frac{1}{n} \sum_{i=1}^{n} \sigma_i ( f(Z_i) )| \right] + \mathbb{E}_{\tilde Z^n\sigma} \left[ \sup_{f\in\mathcal{F}} |\frac{1}{n} \sum_{i=1}^{n} \sigma_i ( f(\tilde Z_i) )| \right]\\
 &=2 \mathbb{E}_{Z^n\sigma} \left[ \sup_{f\in\mathcal{F}} |\frac{1}{n} \sum_{i=1}^{n} \sigma_i ( f(Z_i) )| \right]\\
 &= \mathbb{E}_{Z^n} \left[ \mathbb{E}_{\sigma} \left[ \left. \sup_{f\in\mathcal{F}} |\frac{2}{n} \sum_{i=1}^{n} \sigma_i ( f(Z_i) )| \right| Z^n \right] \right]\\
 &=\Rad(\F)
\end{align*}
\end{proof}














\bibliographystyle{scribebibsty}
\bibliography{s782references}

\end{document}
