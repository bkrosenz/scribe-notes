\documentclass[10pt]{article}
% Include statements
\usepackage{graphicx}
\usepackage{amsfonts,amssymb,amsmath,amsthm}
\usepackage[sort]{natbib}
\usepackage[margin=1in,nohead]{geometry}
\usepackage{multirow,rotating,array}
\usepackage{algorithm,algorithmic}
\usepackage{pdfsync}
\usepackage{hyperref}
\hypersetup{backref,colorlinks=true,citecolor=blue,linkcolor=blue,urlcolor=blue}
\renewcommand{\qedsymbol}{$\blacksquare$}
\setlength{\parindent}{0cm}
\setlength{\parskip}{10pt}


% For sequential numbering
\newcounter{lecnum}
\renewcommand{\thepage}{\thelecnum\ -\ \arabic{page}}
\renewcommand{\thesection}{\thelecnum.\arabic{section}}
\renewcommand{\theequation}{\thelecnum.\arabic{equation}}
\renewcommand{\thefigure}{\thelecnum.\arabic{figure}}
\renewcommand{\thetable}{\thelecnum.\arabic{table}}



% Theorem environments (\autoref compatible)
\usepackage{aliascnt}
\newtheorem{theorem}{Theorem}[lecnum]

\newaliascnt{result}{theorem}
\newtheorem{result}[theorem]{Result}
\aliascntresetthe{result}
\providecommand*{\resultautorefname}{Result}
\newaliascnt{lemma}{theorem}
\newtheorem{lemma}[lemma]{Lemma}
\aliascntresetthe{lemma}
\providecommand*{\lemmaautorefname}{Lemma}
\newaliascnt{prop}{theorem}
\newtheorem{proposition}[prop]{Proposition}
\aliascntresetthe{prop}
\providecommand*{\propautorefname}{Proposition}
\newaliascnt{cor}{theorem}
\newtheorem{corollary}[cor]{Corollary}
\aliascntresetthe{cor}
\providecommand*{\corautorefname}{Corollary}
\newaliascnt{conj}{theorem}
\newtheorem{conjecture}[conj]{Conjecture}
\aliascntresetthe{conj}
\providecommand*{\conjautorefname}{Corollary}
\newaliascnt{def}{theorem}
\newtheorem{definition}[def]{Definition}
\aliascntresetthe{def}
\providecommand*{\defautorefname}{Definition}
\newaliascnt{ex}{theorem}
\newtheorem{example}[ex]{Example}
\aliascntresetthe{ex}
\providecommand*{\exautorefname}{Example}


\newtheorem{assumption}{Assumption}
\renewcommand{\theassumption}{\Alph{assumption}}
\providecommand*{\assumptionautorefname}{Assumption}

\def\algorithmautorefname{Algorithm}
\renewcommand*{\figureautorefname}{Figure}%
\renewcommand*{\tableautorefname}{Table}%
\renewcommand*{\partautorefname}{Part}%
\renewcommand*{\chapterautorefname}{Chapter}%
\renewcommand*{\sectionautorefname}{Section}%
\renewcommand*{\subsectionautorefname}{Section}%
\renewcommand*{\subsubsectionautorefname}{Section}% 


% My Macros
\def\indep{\perp\!\!\!\perp}
\newcommand{\given}{\mbox{ }\vert\mbox{ }}
\newcommand{\F}{\mathcal{F}}
\newcommand{\Expect}[1]{\mathbb{E}\!\left[#1\right]}
\renewcommand{\P}{\mathbb{P}}
\newcommand{\R}{\mathbb{R}}
\newcommand{\X}{\mathcal{X}}
\newcommand{\B}{\mathcal{B}}
\DeclareMathOperator*{\Variance}{Var}
\newcommand{\Var}[1]{\Variance\!\left[#1\right]}
\DeclareMathOperator*{\Covariance}{Cov}
\newcommand{\Cov}[1]{\Covariance\!\left[#1\right]}
\newcommand{\Y}{\mathcal{Y}}
\newcommand{\norm}[1]{\left\lVert #1 \right\rVert}
\newcommand{\email}[1]{\href{mailto:#1}{#1}}
\DeclareMathOperator*{\argmin}{argmin}
\DeclareMathOperator*{\argmax}{argmax}
\newcommand{\indicator}{\mathbbm{1}}
\newcommand{\cdist}{\rightsquigarrow}
\newcommand{\cprob}{\xrightarrow{P}}
\newcommand{\clp}{\xrightarrow{L_p}}
\newcommand{\cas}{\xrightarrow{as}}
\renewcommand{\bar}{\overline}
\renewcommand{\hat}{\widehat}

% Ben's macros:
\renewcommand{\O}{\ensuremath{\mathcal{O}}}
\newcommand{\st}{\ensuremath{\;\mathrm{s.}\;\mathrm{t.}\;}}
\newcommand{\wrt}{\ensuremath{\;\mathrm{w.}\;\mathrm{r.}\;\mathrm{t.}\;}}
\newcommand{\then}{\ensuremath{\;\Rightarrow\;}}


% To be entered
\setcounter{lecnum}{9}
\newcommand{\lecturer}{Prof.\ McDonald}
\newcommand{\scribe}{Ben Rosenzweig}
\newcommand{\chtitle}{Finite Class Lemma and VC Dimension}
\newcommand{\lecdate}{24 October 2017}


\begin{document}
\rule{6.5in}{1pt}

\textsc{STAT--S 782
        \hfill \thelecnum\ --- \chtitle
        \hfill \lecdate}

\textsc{Lecturer: \lecturer \hfill Scribe: \scribe}
\rule{6.5in}{1pt}

\section{Finite Class Lemma}
Last time: control
\rad_n(\F) \then control R(\fhat)

\item concentration of \Delta_n
\item \E

  Today, other ways of bounding rad comp

  \begin{lemma}[finite class (v1)]
    If $\mathcal{A}=\{a^{(1)},...,a^{(N)}\}\subset\R^n$ is a finite set with $||a^{(i)}||\leq L \forall i$ and $N\geq2$, then
    \[
    \rad_n(\mathcal{A})\leq \frac{4L\sqrt{\log{N}}}{n}
   \]
  \end{lemma}
  \begin{proof}
    By some tricks for sub-Gaussian rvs as in ex last wk,
    Y_j = 2/n \sum_i=1^n \sigma_ia_i^{(j)}
    \E{e^{tY_j}}\leq \exp{2L^2t^2/n^2}
    and 
    \E{e^{-tY_j}}\leq \exp{2L^2t^2/n^2},
    so \E{\max_j|Y_j|}=\E{\max{Y_1,-Y_1,...,Y_N,-Y_N}}
    \leq 2L\sqrt{log2N}/n  (here N\geq2, 2N\leq N^2)
    \leq 4L\sqrt{\log N}/n

    Implication: w.p. >1-\delta
    R_n(a^{(j)}\leq \Rhat_n(a^{(j)}+8L\sqrt{\log N}/n+\sqrt{\frac{\log 1/\delta}{n}}
  \end{proof}

  V2 doesnt use rad comp
\begin{lemma}[finite class (v2)]
  P(\sup_j|\rhat_n(a^{(j)}) - R_n(a^{(j)})|\geq\epsilon)
  \leq \sum_j=1^N P(|\rhat_n(a^{(j)})-R_n(a^{(j)})|\geq\epsilon)
  \leq 2Ne^{-n\epsilon^2/(2L)^2}
  w.p. >1-\delta,
  R_n(a^{(j)})\leq \rhat_n(a^{(j)})+2L\sqrt{\frac{2\log N + \log 1/\delta}{n}}
  
  \end{lemma}

how can we extend this to infinite classes
recall \F is projected onto data z^n:
\F(z^n) = \{f(z_1),...,f(z_n):f\in\F\}
this is the set whose size we want to find.
Supp
f \mapsto \{0,1\}\fofrall f\in\F.

then \F(z^n)\subseteq\{0,1\}^n

thus, \forall f\in\F, \sqrt{\sum_i=1^n|f(z_i)|^2}\leq\sqrt{n}

So if z^n is fied,
N:=|\F(z^n)|\leq 2^n (vertex on binary hypercube)
L\leq\sqrt{n}

so \radhat_n(\F(z^n))\leq4\sqrt{\log|\F(z^n)/n}

how can we get a tighter bound?

\section{VC-dimension}
\begin{definition}[Shattering]
  Let \mathcal{C} be a class of subsets of Z.  We say that S=\{z_1,...,z_n\}\subset\Z (finite) 
  is \textit{shattered} by \C if \forall S'\subseteq S
  \exists C\in \C\st S'=S\cap C,
  we say C ``picks out'' S'.
An equivalent definition
\iff
S is shattered by \C if \forall b\in \{0,1}^n\exists C\in \C\st (I(z_1\in C),...,I(z_n\in C))=b
\iff \{(I(z_1\in C),...,I(z_n\in C)):C\in\C\} = \{0,1}^n
  \end{definition}
  
  example class \C
  \C=\{(-\infty,t]:t\in\R\} half-open
  \C=\{(a,b):a\leq b} open
  \C=\{(a,b)\cup(c,d):a\leq b\leq c\leq d} unions
\C= all discs in \R^d
= all axis parallel rectangles in \R^d
= \{x:\beta^Tx\geq0\} (half spaces)
=all convex sets

\begin{example}
\C=\{(a,b):a\leq b}, S=\{1,2,3\}
which S'\subset S can \C pick out?
all S'\in\pow(S) EXCEPT \{1,3\}.  So S is not shattered by \C.
\end{example}

\begin{example}
\C can shatter any set of size two:
\C=\{(a,b):a\leq b}, S=\{s,t\} s\neq t,
but only trivial sets of size 3.
\end{example}    
  
\begin{definition}[Vapnik-Chervonenkis (VC) dimension of \C]
V(\C) = \max{n\in\N:\exists S\subset\Z\st |S|=n and S is shattered by \C}
\end{definition}

(hard part is proving that no set of size > n can be shattered!)

If V(\C)<\infty we say \C is a VC-class

\bdef[the n^th shatter coefficient]
\mathcal{S}_n\C:=\sup_{S\subset\Z,|S|=n}|\{S\cap C:C\in\C\}|
\edef

For \C=\{(a,b):a\leq b},
\S_2(\C)=4
\S_3(\C)=7

another def V(\C) = \max{n\in\N:\S_n(\C)=2^n}

VC is well-def'd: If \S_n<2^n, then \S_m<2^m\forall m>n

VC-dim for binary functions:
\forall f:\Z\to\{0,1\},  let C_f=\{z:f(z)=1\}
\forall C\subseteq\Z, let f_C(x)=I(x\in C)

\begin{definition}
Let \F be a class of f:\Z\to\{0,1\}.  We say S is shattered by \F if S is shattered by C_\F = \{I(f=1):f\in\F\}

where I(f=1) is the indicator of C_f
\S_n(\F) =\S_n(\C_\F)
V(\F)=V(\C_\F)
\edef

\begin{example}
\C = \{(-\infty,t):t\in\R\}
V(\C) = 1
1. shatter some 1-pt set.  S=\{0\}
(-\infty,-1]\cap S = \emptyset
(-\infty,-1]\cap S = S
so we know V(\C)\geq 1
to show there is no 2-pt set,
let S=\{a,b\}, a<b.
Obviously, \not\exists t\st(-\infty,t]=\{b\}.
\end{example}

1. S=\{s,t\} s<t.  choose a_1<a_2<s<a_3<t<a_4
(a_1,a_2)\cap S=\emptyset
(a_2,a_3)\cap S=\{s\}
(a_1,a_4)\cap S=\{t\}
2. S=\{s,t,u\} (s<t<u)
\{s,u\}\subset(a_1,a_2)\then \{t\}\subset(a_1,a_2)

algorithmic stability, bound on rad comp (which you can't use directly for real-valued case unless loss is bdd)

params of k-SVM infinite, but # support vecs finite = VC-dim 

\end{document}
