\documentclass[10pt]{article}
% Include statements
\usepackage{graphicx}
\usepackage{amsfonts,amssymb,amsmath,amsthm}
\usepackage[sort]{natbib}
\usepackage[margin=1in,nohead]{geometry}
\usepackage{multirow,rotating,array}
\usepackage{algorithm,algorithmic}
\usepackage{pdfsync}
\usepackage{hyperref}
\hypersetup{backref,colorlinks=true,citecolor=blue,linkcolor=blue,urlcolor=blue}
\renewcommand{\qedsymbol}{$\blacksquare$}
\setlength{\parindent}{0cm}
\setlength{\parskip}{10pt}

\usepackage[margin=1in]{geometry} 
\usepackage{amsfonts, amsmath,amsthm,amssymb}
\usepackage{mathrsfs}


% For sequential numbering
\newcounter{lecnum}
\renewcommand{\thepage}{\thelecnum\ -\ \arabic{page}}
\renewcommand{\thesection}{\thelecnum.\arabic{section}}
\renewcommand{\theequation}{\thelecnum.\arabic{equation}}
\renewcommand{\thefigure}{\thelecnum.\arabic{figure}}
\renewcommand{\thetable}{\thelecnum.\arabic{table}}



% Theorem environments (\autoref compatible)
\usepackage{aliascnt}
\newtheorem{theorem}{Theorem}[lecnum]

\newaliascnt{result}{theorem}
\newtheorem{result}[theorem]{Result}
\aliascntresetthe{result}
\providecommand*{\resultautorefname}{Result}
\newaliascnt{lemma}{theorem}
\newtheorem{lemma}[lemma]{Lemma}
\aliascntresetthe{lemma}
\providecommand*{\lemmaautorefname}{Lemma}
\newaliascnt{prop}{theorem}
\newtheorem{proposition}[prop]{Proposition}
\aliascntresetthe{prop}
\providecommand*{\propautorefname}{Proposition}
\newaliascnt{cor}{theorem}
\newtheorem{corollary}[cor]{Corollary}
\aliascntresetthe{cor}
\providecommand*{\corautorefname}{Corollary}
\newaliascnt{conj}{theorem}
\newtheorem{conjecture}[conj]{Conjecture}
\aliascntresetthe{conj}
\providecommand*{\conjautorefname}{Corollary}
\newaliascnt{def}{theorem}
\newtheorem{definition}[def]{Definition}
\aliascntresetthe{def}
\providecommand*{\defautorefname}{Definition}
\newaliascnt{ex}{theorem}
\newtheorem{example}[ex]{Example}
\aliascntresetthe{ex}
\providecommand*{\exautorefname}{Example}


\newtheorem{assumption}{Assumption}
\renewcommand{\theassumption}{\Alph{assumption}}
\providecommand*{\assumptionautorefname}{Assumption}

\def\algorithmautorefname{Algorithm}
\renewcommand*{\figureautorefname}{Figure}%
\renewcommand*{\tableautorefname}{Table}%
\renewcommand*{\partautorefname}{Part}%
\renewcommand*{\chapterautorefname}{Chapter}%
\renewcommand*{\sectionautorefname}{Section}%
\renewcommand*{\subsectionautorefname}{Section}%
\renewcommand*{\subsubsectionautorefname}{Section}% 


% My Macros
\def\indep{\perp\!\!\!\perp}
\newcommand{\given}{\mbox{ }\vert\mbox{ }}
\newcommand{\F}{\mathcal{F}}
\newcommand{\Expect}[1]{\mathbb{E}\!\left[#1\right]}
\renewcommand{\P}{\mathbb{P}}
\newcommand{\R}{\mathbb{R}}
\newcommand{\X}{\mathcal{X}}
\newcommand{\B}{\mathcal{B}}
\DeclareMathOperator*{\Variance}{Var}
\newcommand{\Var}[1]{\Variance\!\left[#1\right]}
\DeclareMathOperator*{\Covariance}{Cov}
\newcommand{\Cov}[1]{\Covariance\!\left[#1\right]}
\newcommand{\Y}{\mathcal{Y}}
\newcommand{\norm}[1]{\left\lVert #1 \right\rVert}
\newcommand{\email}[1]{\href{mailto:#1}{#1}}
\DeclareMathOperator*{\argmin}{argmin}
\DeclareMathOperator*{\argmax}{argmax}
\newcommand{\indicator}{\mathbbm{1}}
\newcommand{\cdist}{\rightsquigarrow}
\newcommand{\cprob}{\xrightarrow{P}}
\newcommand{\clp}{\xrightarrow{L_p}}
\newcommand{\cas}{\xrightarrow{as}}
\renewcommand{\bar}{\overline}
\renewcommand{\hat}{\widehat}

% Ben's macros:
\renewcommand{\O}{\ensuremath{\mathcal{O}}}
\newcommand{\st}{\ensuremath{\;\mathrm{s.}\;\mathrm{t.}\;}}
\newcommand{\wrt}{\ensuremath{\;\mathrm{w.}\;\mathrm{r.}\;\mathrm{t.}\;}}
\newcommand{\then}{\ensuremath{\;\Rightarrow\;}}


% To be entered
\setcounter{lecnum}{13}
\newcommand{\lecturer}{Prof.\ McDonald}
\newcommand{\scribe}{Inhak Hwang}
\newcommand{\chtitle}{Specialized techniques}
\newcommand{\lecdate}{12 October 2017}


\begin{document}
\rule{6.5in}{1pt}

\textsc{STAT--S 782
        \hfill \thelecnum\ --- \chtitle
        \hfill \lecdate}

\textsc{Lecturer: \lecturer \hfill Scribe: \scribe}
\rule{6.5in}{1pt}

\section{Histograms}
Suppose $X_{1}...X_{n}$ are i.i.d p on $[0,1]^{d}$. Divide the unit cube into $B_{1},...,B_{n}$ cubes each with side length h, $N \le (\frac{1}{h})^{d}$. 
Define 
\begin{align*}
&\hat{p}_{h}(x) = \sum\limits_{j}\frac{\hat{\theta}_{j}}{h^{d}}I(x \in B_{j}) \text{\hspace{0.5cm}  where  \hspace{0.5cm}} \hat{\theta}_{j} = \frac{1}{n}\sum_{i=1}^{n}I(x_{i} \in B_{j})
\end{align*}
We want to bound
\begin{align*}
\P\left(\| \hat{p}_{h} - p \|_{\infty} > \epsilon \right) 
\end{align*}
Consider :
\begin{align*}
&\P\left(\max\limits_{j}\left|\frac{\hat{\theta}_{j}}{h^{d}} - \frac{\theta_{j}}{h^{d}}\right| > \epsilon\right) 
 & \left(\theta_{j} = \int_{B_{j}}p(x)dx \right) \\
&= \P\left(\max\limits_{j}\left|\hat{\theta}_{j} - \theta_{j}\right| > \epsilon h^{d}\right) 
\end{align*}
assume $|p(x) - p(y) | \le L\|x - y\|$ then $p$ is bounded. 
It follows that
\begin{align*}
\theta_{j} = \int_{B_{j}}p(u)du \le c\int_{B_{j}}du = ch^{d} \\
\rightarrow \theta_{j}(1 - \theta_{j}) \le \theta_{j} \le ch^{d}
\end{align*}
Now
\begin{align*}
\P\left(\left|\hat{\theta}_{j} - \theta_{j}\right| > h^{d}\epsilon\right) &\le 2\exp\left(-\frac{1}{2}\frac{n\epsilon^{2}h^{2d}}{\theta_{j}(1-\theta_{j}) + \epsilon h^{d} / 3}\right) \\
&\le 2\exp\left(-\frac{1}{2}\frac{n\epsilon^{2}h^{2d}}{ch^{d}+ \epsilon h^{d} /3}\right) \\
&\le 2\exp\left(-c n \epsilon^{2} h^{d}\right) 
& \left(c \rightarrow \frac{1}{2(c+1/3)}\right) \\
\end{align*}
\begin{align*}
\P\left(\max\limits_{j}|\hat{\theta}_{j} - \theta_{j}| > h^{d}\epsilon\right) \le 2 h^{-d}\exp\left(-cn\epsilon^{2}h^{d}\right) =: \Pi_{n}
\end{align*}
Now 
\begin{align*}
p(x) - p_{n}(x) = p(x) - \frac{\int_{B_{j}}p(u)du } {h^{d}} \\
                     = \frac{1}{h^{d}}\int_{B_{j}}(p(x) - p(u))du \\
\rightarrow |\cdot | \le \frac{1}{h^{d}}\int_{B_{j}}|p(x)-p(u)|du \\
\le \frac{1}{h^{d}}Lh\sqrt{d}\int du = Lh\sqrt{d}
\end{align*}
wp $1- \Pi_{n}$
\begin{align*}
\|\hat{p}_{h} - p \|_{\infty} \le \|\hat{p}_{h} - p_{h}\|_{\infty} + \| p_{h} - p \|_{\infty}\\ \le \epsilon + L\sqrt{d} h.
\end{align*}
Therefore, wp at least $1- \delta$,
\begin{align*}
\|\hat{p}_{h} - p\|_{\infty} \le \sqrt{\frac{1}{cnh^{d}}\log\left(\frac{2}{\delta h^{d}}\right)} + L\sqrt{d} h.
\end{align*}
Choosing $h = (\frac{c}{n})^{\frac{1}{2+d}}$
gives that,
wp at least $1 - \delta$,
\begin{align*}
\|\hat{p}_{h} - p \|_{\infty} = O\left(\left[\frac{\log n}{n}\right]^{\frac{1}{2+d}}\right)
\end{align*}

\section{Martingales}
Let $(\Omega, \mathscr{F})$. Define sequence $\mathscr{F}_{i}$ of $\sigma$-algebras s.t. $\mathscr{F}_{i} \subset \mathscr{F}_{i+1}$ and $\mathscr{F}_{0} = \sigma\{ \Omega\}$. We call $((\mathscr{F}_{i}, (X_{i}))$ a martingale w.r.t $(\mathscr{F}_{i})$ if $X_{i}$ is $\mathscr{F}_{i}$ measurable and $E[X_{i} | \mathscr{F}_{i-1}] = X_{i-1}$ where $\mathscr{F}_{i} = \sigma(X_{1},X_{2},...,X_{i})$. Note the following 
\begin{itemize}
\item $ E[X_{i}] = \mu $
\item If $E[X_{i}|\mathscr{F}_{i-1}] = 0$ call it a martingale difference.
\end{itemize}

\begin{example}
$Y_{0} = E[X_{0}], Y_{1} = X_{i} - X_{i-1}$
\end{example}
\begin{example}
$\exists X$ s.t. can create $\mathscr{F}_{i} \subset \sigma[X]$, $X_{i} = E[X | \mathscr{F}_{i}$.
\end{example}

\section{Azuma's inequality}
Let $((\mathscr{F}_{i}), (y_{i}))$ be a martingale difference s.t. for $(a_{i})$ and $(b_{i})$ , $P(Y_{i} \in [a_{i}, b_{i}]) = 1 $ for all $i$. Then for $\delta > 0$, 
\begin{align*}
\P\left(\frac{1}{n}\left|\sum Y_{i}\right| \ge \delta\right) \le 2\exp\left\{-\frac{n\delta^{2}}{\sum_{i=1}^{n}(b_{i}-a_{i})^{2}  }\right\}
\end{align*}
\begin{proof} (sketch) \par
$E\left[e^{\lambda \sum_{i=1}^{n}Y_{i}}\right] = E\left[E\left[e^{\lambda\sum_{i=1}^{n}Y_{i}}\given\mathscr{F}_{n-1}\right]\right] $
= $ E\left[e^{\lambda \sum_{i=1}^{n-1} Y_{i}} E\left[e^{ \lambda Y_{n} } \given \mathscr{F}_{n-1}\right]\right] $ 
\end{proof}


\begin{theorem}[McDiamird's]
Let $Z = f(X_{1},...X_{n})$. If $f$ satisfies ``bounded differences" : \\
\begin{align*}
\sup\limits_{x_{1},x_{2}...x_{n},x'\in X^{n+1}}|f(x_{1},...,x_{n}) - f(x_{2},..., x_{i-1},x', x_{i+1},...,x_{n})| < c_{i}
\end{align*}
Then $\delta > 0$, $P(Z - E[Z] > \delta) \le \exp\left(-\frac{2\delta}{\sum c_{i}^{2}}\right)$
\end{theorem}
\begin{proof} (sketch) \\
Take
$Y_{0} = E[Z]$, $Y_{1} = E[Z|X_{1}...X_{i}] = E[Z|X_{1}...X_{i-1}]$ and use Azuma.
\end{proof}
\bibliographystyle{scribebibsty}
\bibliography{s782references}

\end{document}