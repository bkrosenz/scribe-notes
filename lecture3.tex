\documentclass[10pt]{article}
% Include statements
\usepackage{graphicx}
\usepackage{amsfonts,amssymb,amsmath,amsthm}
\usepackage[sort]{natbib}
\usepackage[margin=1in,nohead]{geometry}
\usepackage{multirow,rotating,array}
\usepackage{algorithm,algorithmic}
\usepackage{pdfsync}
\usepackage{hyperref}
\usepackage{braket}
\hypersetup{backref,colorlinks=true,citecolor=blue,linkcolor=blue,urlcolor=blue}
\renewcommand{\qedsymbol}{$\blacksquare$}
\setlength{\parindent}{0cm}
\setlength{\parskip}{10pt}


% For sequential numbering
\newcounter{lecnum}
\renewcommand{\thepage}{\thelecnum\ -\ \arabic{page}}
\renewcommand{\thesection}{\thelecnum.\arabic{section}}
\renewcommand{\theequation}{\thelecnum.\arabic{equation}}
\renewcommand{\thefigure}{\thelecnum.\arabic{figure}}
\renewcommand{\thetable}{\thelecnum.\arabic{table}}



% Theorem environments (\autoref compatible)
\usepackage{aliascnt}
\newtheorem{theorem}{Theorem}[lecnum]

\newaliascnt{result}{theorem}
\newtheorem{result}[theorem]{Result}
\aliascntresetthe{result}
\providecommand*{\resultautorefname}{Result}
\newaliascnt{lemma}{theorem}
\newtheorem{lemma}[lemma]{Lemma}
\aliascntresetthe{lemma}
\providecommand*{\lemmaautorefname}{Lemma}
\newaliascnt{prop}{theorem}
\newtheorem{proposition}[prop]{Proposition}
\aliascntresetthe{prop}
\providecommand*{\propautorefname}{Proposition}
\newaliascnt{cor}{theorem}
\newtheorem{corollary}[cor]{Corollary}
\aliascntresetthe{cor}
\providecommand*{\corautorefname}{Corollary}
\newaliascnt{conj}{theorem}
\newtheorem{conjecture}[conj]{Conjecture}
\aliascntresetthe{conj}
\providecommand*{\conjautorefname}{Corollary}
\newaliascnt{def}{theorem}
\newtheorem{definition}[def]{Definition}
\aliascntresetthe{def}
\providecommand*{\defautorefname}{Definition}
\newaliascnt{ex}{theorem}
\newtheorem{example}[ex]{Example}
\aliascntresetthe{ex}
\providecommand*{\exautorefname}{Example}


\newtheorem{assumption}{Assumption}
\renewcommand{\theassumption}{\Alph{assumption}}
\providecommand*{\assumptionautorefname}{Assumption}

\def\algorithmautorefname{Algorithm}
\renewcommand*{\figureautorefname}{Figure}%
\renewcommand*{\tableautorefname}{Table}%
\renewcommand*{\partautorefname}{Part}%
\renewcommand*{\chapterautorefname}{Chapter}%
\renewcommand*{\sectionautorefname}{Section}%
\renewcommand*{\subsectionautorefname}{Section}%
\renewcommand*{\subsubsectionautorefname}{Section}% 


% My Macros
\def\indep{\perp\!\!\!\perp}
\newcommand{\given}{\mbox{ }\vert\mbox{ }}
\newcommand{\F}{\mathcal{F}}
\newcommand{\Expect}[1]{\mathbb{E}\!\left[#1\right]}
\renewcommand{\P}{\mathbb{P}}
\newcommand{\R}{\mathbb{R}}
\newcommand{\X}{\mathcal{X}}
\newcommand{\B}{\mathcal{B}}
\DeclareMathOperator*{\Variance}{Var}
\newcommand{\Var}[1]{\Variance\!\left[#1\right]}
\DeclareMathOperator*{\Covariance}{Cov}
\newcommand{\Cov}[1]{\Covariance\!\left[#1\right]}
\newcommand{\Y}{\mathcal{Y}}
\newcommand{\norm}[1]{\left\lVert #1 \right\rVert}
\newcommand{\email}[1]{\href{mailto:#1}{#1}}
\DeclareMathOperator*{\argmin}{argmin}
\DeclareMathOperator*{\argmax}{argmax}
\newcommand{\indicator}{\mathbbm{1}}
\newcommand{\cdist}{\rightsquigarrow}
\newcommand{\cprob}{\xrightarrow{P}}
\newcommand{\clp}{\xrightarrow{L_p}}
\newcommand{\cas}{\xrightarrow{as}}
\renewcommand{\bar}{\overline}
\renewcommand{\hat}{\widehat}


% Your new macros

\newcommand{\Rn}{\R^n}
\newcommand{\sumi}[2]{\sum\limits_{#1=1}^{#2}}
\newcommand{\TODO}[1]{\textbf{\color{red} TODO: #1}}
\newcommand{\trans}[1]{{#1}^\mathsf{T}}
\newcommand{\Sym}{\mathbb{S}^n}
\DeclareMathOperator*{\dom}{dom}


%%%%%%%%%%%%%%%%%%%%%%%%%%%%%%%%
% To be entered
\setcounter{lecnum}{3}
\newcommand{\lecturer}{Prof.\ McDonald}
\newcommand{\scribe}{Dmitrii Avdiukhin}
\newcommand{\chtitle}{Convex sets and functions}
\newcommand{\lecdate}{30 August 2017}


\begin{document}
\rule{6.5in}{1pt}

\textsc{STAT--S 782
        \hfill \thelecnum\ --- \chtitle
        \hfill \lecdate}

\textsc{Lecturer: \lecturer \hfill Scribe: \scribe}
\rule{6.5in}{1pt}



\section{Convex functions}
\label{sec:convex-functions}

\begin{definition}
  Set $C$ is \emph{convex} iff \ \ $\forall x,c \in C, \forall t \in [0; 1]\ \ t x + (1-t) y \in C.$
\end{definition}
So $C$ is convex iff for any two points in $C$ their segment is also entirely in $C$.
%\TODO{picture}
\begin{definition}
  \emph{Convex combination} of set of points $x_1, \ldots, x_k \in \Rn$ is
  \[\Set{\sumi ik \Theta_i x_i | \sumi ik \Theta_i=1,\ \forall i\ \Theta_i \in [0; 1]}.\]
\end{definition}

\begin{definition}
  \emph{Convex hull} of any $C \in \Rn$, denoted $conv(C)$ is a union of all convex combinations of different elements of $C$.
\end{definition}

Examples:
\begin{itemize}
  \item Empty set, point, line, segment.
  \item Norm ball: $\Set{x | \norm x < r}$.
  \item Hyperplane $\Set{x | \trans{a} x = b}$, Affine space $\Set{x | A x = b}$.
  \item Hyperspace: $\Set{x | \trans{a} x \leq b}$, Polyhedron $\Set{x | A x \leq b}$.
  \item Cone such that if $x_1, x_2 \in C$ then $t_1 x_1 + t_2 x_2 \in C \ \forall t_1, t_2 \geq 0$.
\end{itemize}
\begin{definition}
  Set $C$ is a \emph{cone} iff \ \ $\forall t \geq 0,\ x \in C \implies t x \in C$.
\end{definition}
Type of cones:
\begin{itemize}
  \item Norm cone: $\Set{(x,t) | \norm x \leq t}$.
  \item Normal cone for some $C$ and $x \in C$: $N_C(x) = \Set{g | \trans g x \geq \trans g y \ \forall y \in C}$.
  \item Positive semidefinite cone $S_+^n = \Set{x \in S^n | x \succeq 0}$, $S^n$ is Hilbert space.
\end{itemize}

\section{Key properties}
%\TODO{pictures}
\begin{itemize}
  \item Separation hyperplane. $A, B$ are convex, nonempty, disjoint. Then
    $\exists a,b:\ A \subseteq \Set{x | \trans a \leq b}, B \subseteq \Set{x | \trans a \geq b}$.
  \item Supporting hyperplane. $C$ nonempty, convex, $x_0 \in boundary(C)$. Then
    $\exists a:\ C \subseteq \Set{x | \trans a x \leq \trans a x_0}$.
\end{itemize}


\section{Operations preserving convexity}
\begin{itemize}
  \item Intersection.
  \item Scaling, translation. $C$ is convex $\implies a C + b$ is convex.
  \item Affine image and preimage. $f(x) = A x + b$, $C$ is convex $\implies f(C), f^{-1}(C)$ are convex.
  \item Lots more (See \cite{boyd2004convex}, chapter $2$).
\end{itemize}

\begin{example}
  $A_1, \ldots, A_k, B \in \Sym$ -- symmetrical matrices.
  Then $C = \Set{x \in \R^k | \sumi ik x_i A_i \preceq B}$.
\end{example}
\begin{proof}
  $f: \R^k \to \Sym$, $f(x) = B - \sumi i k x_i A_i$.
  $C = f^{-1}(S_+^n)$ -- affine preimage of convex cone.
\end{proof}

\section{Convex function}

\begin{definition}
  Function $f: \Rn \to \R$ is \emph{convex} iff $\dom f \subseteq \Rn$ is convex and
    \[\forall x,y \in \dom f, t \in [0;1] \ \ \ f(t x + (1-t) y) \leq t f(x) + (1-t) f(y)\]
\end{definition}

Other definitions:
\begin{itemize}
  \item $f$ is \emph{concave} iff $-f$ is convex.
  \item $f$ is \emph{strictly convex} iff $\forall t \in (0; 1)$ the inequality in definition is strict.
  \item $f$ is \emph{strongly convex} with parameter $\tau$
    iff $f(x) - \frac \tau 2 \norm x_2^2$ is convex.
\end{itemize}

Examples:
\begin{itemize}
  \item $f(x) = \frac 1 x$ is strictly convex, but not strongly.
  \item Univariate functions:
    \begin{itemize}
      \item $e^{ax}$ is convex $\forall a \in \R$ over $\R$.
      \item $x^a$ convex given $a \geq 1$ or $a \leq 1$ over $\R_+$.
      \item $\log x$ is concave over $\R_+$.
    \end{itemize}
  \item Affine $\trans a x b$ is both convex and concave.
  \item Quadratic $\frac 1 2 \trans x Q x + \trans b x + c$ is convex if $Q \succeq 0$.
  \item \ \ \ \ $\norm {u - A x}_2^2$ convex since $\trans A A \succeq 0$.
  \item Norms: all vector norms and most matrix norms are convex.
  \item Indicator function is convex. $C$ is a convex set, then
    $I_C(x) = \begin{cases}
                0,\ x \in C \\
                \infty, otherwise
              \end{cases}$.
  \item Support function is convex $\forall C$. $I_C^*(x) = \max\limits_{y \in C} \trans x y$.
\end{itemize}

\section{Key properties}
\begin{itemize}
  \item $f$ is convex iff its epigraph is convex, where
    $epi(f) = \set{ (x,t) \in \dom f \times R | f(x) \leq t}$.
  \item $f$ is convex $\implies$ all its sublevel sets are convex. $C_t = \set{x \in \dom f | f(x) \leq t}$.
    The converse is false.
  \item Assume $f$ is differentiable. Then $f$ is convex iff $\dom f$ is convex and
    $\forall x,y \in \dom f\ \ f(y) \geq f(x) + \nabla \trans{f(x)} (y-x)$.
    Essentially, it means that $f$'s graph is above any tangent plain.
  \item Assume $f$ is twice differentiable.
    $f$ is convex iff $\dom f$ is convex and $\forall x \in \dom f\ \ \nabla^2 f(x) \succeq 0$.
\end{itemize}

\section{Operations preserving function convexity}

\begin{itemize}
  \item Nonnegative linear combination.
  \item Pointwise maximum.
    $\forall s \in S\ \ f_s$ is convex $\implies f(x) = \max\limits_S f_s(x)$ is also convex.
  \item Partial minimum. $g(x,y)$ convex over variables $x, y$; $C$ convex.
    Then $f(x) = \min\limits_{y \in C} g(x,y)$ is also convex.
    E.g., $f(x) = \max\limits_{y \in C} \norm{x-y}$ or $f(x) = \min\limits_{y \in C} \norm{x-y}$.
\end{itemize}

\bibliographystyle{scribebibsty}
\bibliography{s782references}

\end{document}
