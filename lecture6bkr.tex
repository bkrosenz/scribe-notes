\documentclass[10pt]{article}
% Include statements
\usepackage{graphicx}
\usepackage{amsfonts,amssymb,amsmath,amsthm}
\usepackage[sort]{natbib}
\usepackage[margin=1in,nohead]{geometry}
\usepackage{multirow,rotating,array}
\usepackage{algorithm,algorithmic}
\usepackage{pdfsync}
\usepackage{hyperref}
\hypersetup{backref,colorlinks=true,citecolor=blue,linkcolor=blue,urlcolor=blue}
\renewcommand{\qedsymbol}{$\blacksquare$}
\setlength{\parindent}{0cm}
\setlength{\parskip}{10pt}


% For sequential numbering
\newcounter{lecnum}
\renewcommand{\thepage}{\thelecnum\ -\ \arabic{page}}
\renewcommand{\thesection}{\thelecnum.\arabic{section}}
\renewcommand{\theequation}{\thelecnum.\arabic{equation}}
\renewcommand{\thefigure}{\thelecnum.\arabic{figure}}
\renewcommand{\thetable}{\thelecnum.\arabic{table}}



% Theorem environments (\autoref compatible)
\usepackage{aliascnt}
\newtheorem{theorem}{Theorem}[lecnum]

\newaliascnt{result}{theorem}
\newtheorem{result}[theorem]{Result}
\aliascntresetthe{result}
\providecommand*{\resultautorefname}{Result}
\newaliascnt{lemma}{theorem}
\newtheorem{lemma}[lemma]{Lemma}
\aliascntresetthe{lemma}
\providecommand*{\lemmaautorefname}{Lemma}
\newaliascnt{prop}{theorem}
\newtheorem{proposition}[prop]{Proposition}
\aliascntresetthe{prop}
\providecommand*{\propautorefname}{Proposition}
\newaliascnt{cor}{theorem}
\newtheorem{corollary}[cor]{Corollary}
\aliascntresetthe{cor}
\providecommand*{\corautorefname}{Corollary}
\newaliascnt{conj}{theorem}
\newtheorem{conjecture}[conj]{Conjecture}
\aliascntresetthe{conj}
\providecommand*{\conjautorefname}{Corollary}
\newaliascnt{def}{theorem}
\newtheorem{definition}[def]{Definition}
\aliascntresetthe{def}
\providecommand*{\defautorefname}{Definition}
\newaliascnt{ex}{theorem}
\newtheorem{example}[ex]{Example}
\aliascntresetthe{ex}
\providecommand*{\exautorefname}{Example}


\newtheorem{assumption}{Assumption}
\renewcommand{\theassumption}{\Alph{assumption}}
\providecommand*{\assumptionautorefname}{Assumption}

\def\algorithmautorefname{Algorithm}
\renewcommand*{\figureautorefname}{Figure}%
\renewcommand*{\tableautorefname}{Table}%
\renewcommand*{\partautorefname}{Part}%
\renewcommand*{\chapterautorefname}{Chapter}%
\renewcommand*{\sectionautorefname}{Section}%
\renewcommand*{\subsectionautorefname}{Section}%
\renewcommand*{\subsubsectionautorefname}{Section}% 


% My Macros
\def\indep{\perp\!\!\!\perp}
\newcommand{\given}{\mbox{ }\vert\mbox{ }}
\newcommand{\F}{\mathcal{F}}
\newcommand{\Expect}[1]{\mathbb{E}\!\left[#1\right]}
\renewcommand{\P}{\mathbb{P}}
\newcommand{\R}{\mathbb{R}}
\newcommand{\X}{\mathcal{X}}
\newcommand{\B}{\mathcal{B}}
\DeclareMathOperator*{\Variance}{Var}
\newcommand{\Var}[1]{\Variance\!\left[#1\right]}
\DeclareMathOperator*{\Covariance}{Cov}
\newcommand{\Cov}[1]{\Covariance\!\left[#1\right]}
\newcommand{\Y}{\mathcal{Y}}
\newcommand{\norm}[1]{\left\lVert #1 \right\rVert}
\newcommand{\email}[1]{\href{mailto:#1}{#1}}
\DeclareMathOperator*{\argmin}{argmin}
\DeclareMathOperator*{\argmax}{argmax}
\newcommand{\indicator}{\mathbbm{1}}
\newcommand{\cdist}{\rightsquigarrow}
\newcommand{\cprob}{\xrightarrow{P}}
\newcommand{\clp}{\xrightarrow{L_p}}
\newcommand{\cas}{\xrightarrow{as}}
\renewcommand{\bar}{\overline}
\renewcommand{\hat}{\widehat}


% Your new macros



% To be entered
\setcounter{lecnum}{0}
\newcommand{\lecturer}{Prof.\ McDonald}
\newcommand{\scribe}{Prof.\ McDonald}
\newcommand{\chtitle}{Scribe template}
\newcommand{\lecdate}{7 August 2017}


\begin{document}
\rule{6.5in}{1pt}

\textsc{STAT--S 782
        \hfill \thelecnum\ --- \chtitle
        \hfill \lecdate}

\textsc{Lecturer: \lecturer \hfill Scribe: \scribe}
\rule{6.5in}{1pt}

\section{strong duality in LPs}

\begin{enumerate}
\item Dual of dual is primal
\item strong duality if (P) is feasible
\item strong duality if (D) is feasible
\item 2\&3\Rightarrow strong duality unless both (P)\&(D) infeasible
\end{itemize}

\begin{example}
  SVMs
  min_\beta,beta_0,\xi 1/n ||\beta||^2_2+C\sum_i=1^n \xi_i\st \xi_i\geq0, y_i(x_i^T\beta+\beta_0)\geq1-\xi_i

  L(\beta,\beta_0,\xi,v,w)=1/2||\beta||_2^2+C1^T\xi-v^T\xi+\sum w_i(1-\xi_i-y_i(x_i^T\beta-\beta_0))

  DUAL:
  max_w -1/2 w^T~{X}~{X}^Tw+1^Tw \st 0\leq w\leq C1, w^Ty=0, w=0 is feasible
\end{example}

KKT conditions

1. stationarity
0\in\partial L
(If P is convex, \partial(\cdot)=\partial f+u^T\partial h+v^T\partial l)

2. complementary slackness

3. primal feasibility

4. dual feasibility

if x*, (u*, v*) are optimal for (P) and (D) and f*=g*, then x*,(u*,v*) satisfy KKT
\begin{proof}
  f(x*)=g(u*,v*)\leq min_x f(x)+u*^Th(x)+v*^Tl(x)

  ...
  \end{proof}

for SVM, KKT don't give solution, but w_i=0 unless 1-\xi_i-y_i(x_i^T\beta+\beta_0)=0, i.e. only the support points matter

for l_1 regularization, KKT tells you some conds can be zero

when are (C) min_xf(x)\st h(x)\leq t and (L) min_xf(x)+\lambda h(x) equiv?

x* solves C\Rightarrow L if:
(C) strictly feasible \rightarrow strong duality (Slater's cond)
\rightarrow \exists \lambda>0\st\forall x that solve C, those x minimize L
\rightarrow x* solves L

x* solves L \rightarrow x* satisfies KKT at h(x)=t \rightarrow solves C


KKT are:
necessary for optimality under strong duality
always sufficient for optimality
duality gap: for x, (u,v) feasible, f(x)-f(x*)\leqf(x)-g(u,v)
under strong duality, for some optimal (u*,v*), any primal soln minimizes the Lagrangian L(x,u*,v*).

\section{Fenchel conjugate}
f,f*:\R^n\to\R
f*(y)=max_x y^Tx-f(x)
always convex
largest gap btwn line y^Tx and f(x)

poperties
Fenchel's ineq: f(x)+f*(y)\geq x^Ty
f**\leq f
f closed \& convex\then f**=f\forall x,y
x\in\partial f*(y)\iff y\in\partial f(x)\iff f(x)+f*(x)=x^Ty
f(u,v)=f1(u)+f2(v) \then f*(w,z)=f1*(w)+f2*(z)
f(x)=ag(x)+b\then f*(y)=ag*(y/a)-b
A\in\R^n\times n nonsingular f(x) = g(Ax+b) \then f*(y)=g*(A^-Ty)-b^TA^-T

Dual Norm:
\ell_q s.t. for \ell_p, H\"{o}lder's ineq holds (1/p+1/q=1)

Trace norm:
(||X||_tr)_*=||X||_op=\sigma_1(x)
||x||_** = ||x||
conjugate = I_{z:||z||_*\leq1}(y)

Squared norm:
f=||x||^2/2 \then f*=||y||_*^2/2

Affine fn
f(x)=a^Tx+b\then f*(y)=max_xy^Tx-a^Tx-b
bounded \iff y=a
f*= -b if y=a else \infty

f(x)=-log(x), dom(f)=\R+
f*= -log(-y)-1 if y<0 else \infty

\begin{example}
  min_x f(x)+g(x) \iff min_x,z f(x)+g(z)\st x=z
  dual is -f*(u)-g*(-u)
  dual of min_x f(x)+||x| is max_u -f*(u)-I_{z:||z|_*\leq 1}(-u)  
  \end{example}

\begin{example}
  Lasso
  min_beta 1/2||y-X\beta||_2^2+\lambda||\beta||_1 \iff min_beta,z 1/2||y-z||_2^2+\lambda||\beta||_1 \st z=X\beta
  Dual 1/2||v||_2^2 is 1/2||v||_2^2
  g(u) = 1/2||u||_2^2+y^Tu--I_{v:||v|_*\leq 1}(x^Tu/\lambda)
  dual prob:
  max_u 1/2||2||_2^2+y^Tu \st ||x^Tu||_\infty\leq\lambda
  \iff min_u 1/2||y-u||_2^2 \st||x^Tu||_\infty\leq\lambda
  strong duality holds, but the optima f* and g* are diff
  KKT stationarity X\beta=y-u
  \end{example}

shift linear transformations
min_x f(x)+h(Ax)\iff min_x,z f(x)+h(z)\st Ax=z
g(u)=-f*(A^Tu)-g*(-u)

\bibliographystyle{scribebibsty}
\bibliography{s782references}

\end{document}
