
\documentclass[10pt]{article}
% Include statements
\usepackage{graphicx}
\usepackage{amsfonts,amssymb,amsmath,amsthm}
\usepackage[sort]{natbib}
\usepackage[margin=1in,nohead]{geometry}
\usepackage{multirow,rotating,array}
\usepackage{algorithm,algorithmic}
\usepackage{pdfsync}
\usepackage{hyperref}
\hypersetup{backref,colorlinks=true,citecolor=blue,linkcolor=blue,urlcolor=blue}
\renewcommand{\qedsymbol}{$\blacksquare$}
\setlength{\parindent}{0cm}
\setlength{\parskip}{10pt}


% For sequential numbering
\newcounter{lecnum}
\renewcommand{\thepage}{\thelecnum\ -\ \arabic{page}}
\renewcommand{\thesection}{\thelecnum.\arabic{section}}
\renewcommand{\theequation}{\thelecnum.\arabic{equation}}
\renewcommand{\thefigure}{\thelecnum.\arabic{figure}}
\renewcommand{\thetable}{\thelecnum.\arabic{table}}



% Theorem environments (\autoref compatible)
\usepackage{aliascnt}
\newtheorem{theorem}{Theorem}[lecnum]

\newaliascnt{result}{theorem}
\newtheorem{result}[theorem]{Result}
\aliascntresetthe{result}
\providecommand*{\resultautorefname}{Result}
\newaliascnt{lemma}{theorem}
\newtheorem{lemma}[lemma]{Lemma}
\aliascntresetthe{lemma}
\providecommand*{\lemmaautorefname}{Lemma}
\newaliascnt{prop}{theorem}
\newtheorem{proposition}[prop]{Proposition}
\aliascntresetthe{prop}
\providecommand*{\propautorefname}{Proposition}
\newaliascnt{cor}{theorem}
\newtheorem{corollary}[cor]{Corollary}
\aliascntresetthe{cor}
\providecommand*{\corautorefname}{Corollary}
\newaliascnt{conj}{theorem}
\newtheorem{conjecture}[conj]{Conjecture}
\aliascntresetthe{conj}
\providecommand*{\conjautorefname}{Corollary}
\newaliascnt{def}{theorem}
\newtheorem{definition}[def]{Definition}
\aliascntresetthe{def}
\providecommand*{\defautorefname}{Definition}
\newaliascnt{ex}{theorem}
\newtheorem{example}[ex]{Example}
\aliascntresetthe{ex}
\providecommand*{\exautorefname}{Example}


\newtheorem{assumption}{Assumption}
\renewcommand{\theassumption}{\Alph{assumption}}
\providecommand*{\assumptionautorefname}{Assumption}

\def\algorithmautorefname{Algorithm}
\renewcommand*{\figureautorefname}{Figure}%
\renewcommand*{\tableautorefname}{Table}%
\renewcommand*{\partautorefname}{Part}%
\renewcommand*{\chapterautorefname}{Chapter}%
\renewcommand*{\sectionautorefname}{Section}%
\renewcommand*{\subsectionautorefname}{Section}%
\renewcommand*{\subsubsectionautorefname}{Section}% 


% My Macros
\def\indep{\perp\!\!\!\perp}
\newcommand{\given}{\mbox{ }\vert\mbox{ }}
\newcommand{\F}{\mathcal{F}}
\newcommand{\Expect}[1]{\mathbb{E}\!\left[#1\right]}
\renewcommand{\P}{\mathbb{P}}
\newcommand{\R}{\mathbb{R}}
\newcommand{\X}{\mathcal{X}}
\newcommand{\B}{\mathcal{B}}
\DeclareMathOperator*{\Variance}{Var}
\newcommand{\Var}[1]{\Variance\!\left[#1\right]}
\DeclareMathOperator*{\Covariance}{Cov}
\newcommand{\Cov}[1]{\Covariance\!\left[#1\right]}
\newcommand{\Y}{\mathcal{Y}}
\newcommand{\norm}[1]{\left\lVert #1 \right\rVert}
\newcommand{\email}[1]{\href{mailto:#1}{#1}}
\DeclareMathOperator*{\argmin}{argmin}
\DeclareMathOperator*{\argmax}{argmax}
\newcommand{\indicator}{\mathbbm{1}}
\newcommand{\cdist}{\rightsquigarrow}
\newcommand{\cprob}{\xrightarrow{P}}
\newcommand{\clp}{\xrightarrow{L_p}}
\newcommand{\cas}{\xrightarrow{as}}
\renewcommand{\bar}{\overline}
\renewcommand{\hat}{\widehat}


% Your new macros



% To be entered
\setcounter{lecnum}{12}
\newcommand{\lecturer}{Prof.\ McDonald}
\newcommand{\scribe}{Adithya Vadapalli}
\newcommand{\chtitle}{Concentration I}
\newcommand{\lecdate}{7 August 2017}


\begin{document}
	\rule{6.5in}{1pt}
	
	\textsc{STAT--S 782
		\hfill \thelecnum\ --- \chtitle
		\hfill \lecdate}
	
	\textsc{Lecturer: \lecturer \hfill Scribe: \scribe}
	\rule{6.5in}{1pt}
	
	
	\section*{Introduction}
	In this lecture we will study sub-Gaussian Random Variables and study their properties.
	
	\section{Sub-Gaussian Random Variables}
	Many important classes of random variables have tail probabilities decreasing at
	least as rapidly as normally distributed random variables. In order to facilitate
	the exploration of this phenomenon, we find it useful to formalize the notion of a
	sub-Gaussian random variable.
	\begin{definition} (Sub Gaussian)
		Let $X$ be a Random Variable, If $\exists \sigma > 0$ such that, $\mathbb{E}[e^{(\lambda x)}] \le e^{\frac{\sigma^2 \lambda^2}{2}}$, then we call $X$ $\sigma$-Sub Gaussian.
	\end{definition}
	We will now show a few examples and prove that, they are indeed sub-Gaussian.
	\begin{example}
		Gaussian Random Variable with mean 0 and variance $\sigma^{2}$ is $\sigma$ sub-Gaussian.
	\end{example}

    \begin{example}(Radamachar Variable)
    	$\P[X = 1] = \P[X = -1] = 1/2$ is sub-Gaussian with $\sigma = 1$
    \end{example}
	
	We will prove that Radamachar Variable is indeed $1$-Sub Gaussian. 
	\begin{proof}
		\begin{align*} 
		\mathbb{E}[e^{\lambda X}] &= e^{-\lambda} \P[X = -1] + e^{\lambda} \P[X = 1] \\ 
		 &=  -1/2(e^{-\lambda} + e^{\lambda}) \\
		 &= \cosh(\lambda) \\
		 &= \sum_{k=0}^{\infty} \frac{\lambda^{2k}}{(2k)!} \\
		 &\le \sum_{k=0}^{\infty} \frac{(\lambda^{2})^{k}}{2^{k} k!} \\
		 &= e^{\lambda^{2}/2}
		\end{align*}
		Thus $\sigma = 1$
	\end{proof} 
	
	\begin{example} 
		Uniform Distribution between $[-a, a]$ is $a$-sub Gaussian.	
	\end{example}
	
	We will prove this.
	\begin{proof}
		\begin{align*} 
		\mathbb{E}[e^{\lambda X}] &=  -1/2a(e^{-\lambda x} + e^{\lambda x}) \\
		&= \sinh(a\lambda) \\
		&= \sum_{k=0}^{\infty} \frac{a \lambda^{2k}}{(2k + 1)!} \\
		&= e^{a^{2}\lambda^{2}/2}
		\end{align*}
		Thus $\sigma = a$.
	\end{proof}
	
	\section{Properties of Sub-Gaussian Random Variables}
	In this section we will provide a few properties of the Sub-Gaussian Random Variables. We will not prove all the results.
	\begin{theorem} Following statements hold true.
		\begin{enumerate}
			\item 	If $X_1$ is $\sigma_1$-sub Gaussian, then $aX_1$ is $|a|\sigma_1$ sub-Gaussian.
			\item If $X_1$ is $\sigma_1$-sub Gaussian and $X_2$ is $\sigma_2$-sub Gaussian, then $X_1 + X_2$ is $(\sigma_1 + \sigma_2)$ sub Gaussian.
			
			\item  If $X_1$ is $\sigma_1$-sub Gaussian and $X_2$ is $\sigma_2$-sub Gaussian, and $X_1$ and $X_2$ are independent then - $X_1 + X_2$ is $\sqrt{\sigma_1^{2} + \sigma_2^{2}}$- sub Gaussian.
		\end{enumerate}
	
	\begin{proof}. 
		\begin{enumerate}
			\item $\mathbb{E}[e^{\lambda a X}] \le e^{\lambda^2 a^2 \sigma^2 /2}$.
			\item Follows from Independence.
			\item
			\begin{align*}
			\mathbb{E}[e^{\lambda(X_1 + X_2)}] &\le \mathbb{E}[e^{\lambda X_1 p}]^{1/p} \mathbb{E}[e^{\lambda X_2 q}]^{1/q} \\
							&\le (e^{\sigma_1^2/2 \lambda^2 p^2})^{1/p} (e^{\sigma_1^2/2 \lambda^2 q^2})^{1/q}\\
							&= e^{\lambda^2/2 (p \sigma_1^2 + q \sigma_2^{2} )}\\
							&= e^{\lambda^2 (\sigma_1 +  \sigma_2)^2/2}
			\end{align*}

		\end{enumerate}
	\end{proof}
	
	\end{theorem}

 \begin{theorem}
 	Suppose $\mathbb{E}[X = 0]$, then following are equivalent.
 	\begin{enumerate}
 	 \item  $\P[X > \delta]$ \textbf{OR} $\P[-X \ge \delta] \le e^{-\delta^{2}/2\sigma^{2}}$
 	 \item  For every integer $q \ge 1$, $\mathbb{E}[X^{2q}] \le 2q!(2\sigma^2)^q \le q! (4 \sigma^2)^q$
 
 	 \item $\exists \alpha$ such that $\mathbb{E}[e^{\alpha X^2}] \le 2$.
 	 \item $\mathbb{E}[e^{\lambda X}] \le e^{\sigma^2 \lambda^2 / 2}$ for some $\sigma^2$
	\end{enumerate}	
 \end{theorem}
\begin{proof}
	$4 \implies 1$ due to Chernoff Bound. $3 \implies 4$ and $4 \implies 3$ for $\sigma^2 \in [2/\alpha, 4/\alpha]$. We omit the remaining proofs. 
\end{proof}

\section{Bounded Random Variables}
Bounded variables are an important class of sub-Gaussian
random variables. The sub-Gaussian property of bounded random variables is
established by the following lemma:
\begin{lemma}(Hoeffding's Inequality)
	Let $X$ be a Random Variable such that $\mathbb{E}[X] = 0$ and $\P[X \in [a,b]] = 1$, then: $\mathbb{E}[e^{\lambda X}] \le e^{\lambda^2(b - a)^2/8}$
\end{lemma}
\begin{proof}
	Write $X$ as a convex combination $a$ and $b$. Define $\alpha = \frac{X - a}{b - a}$. $X = \alpha b + (1 - \alpha) X$ , $\alpha \in [0,1]$.
	By Convexity we can write:
	\begin{align*}
	 e^{\lambda x} &\le \alpha e^{\lambda b} + (1 - \alpha) e^{\lambda a}\\
	 				&= \frac{X - a}{b - a} e^{\lambda b} + \frac{b - \lambda}{b - a} e^{\lambda a} \\
	 	\mathbb{E}[e^{\lambda x}] &\le \frac{-a}{b - a}e^{\lambda b} + \frac{b}{b - a}e^{\lambda a} \\
	 	 &= e^{g(u)} \quad (u = \lambda(b - a))
	\end{align*}
	
	$g(u) = -\gamma u + \log(1 - \gamma + \gamma e^2)$ where $\gamma = -a/(b - a)$
	
	Note that, 
	\[
	g(0) = g'(0) = 0 \text{and} g''(u) \le 1/4.
	\] 
	By Taylor's Theorem: 
	\[
	\exists x \in (0, u) \text{ \textit{such that} } g(u) = g(0) + u g'(0) + u^{2}/2 g''(u)
	\]
	
	Note that, 
	\[
	u^2/2 g''(u) \le u^{2}/8 = \lambda^2 (b - a)^2 /8
	\] 
	
	Thus we have 
	\[
	\mathbb{E}[e^{-\lambda x}] \le e^{g(u)} \le e^{\frac{\lambda^2 (b - a)^2}{8}}
	\]
	
\end{proof}
	
	\bibliographystyle{scribebibsty}
	\bibliography{s782references}
	
\end{document}

