\documentclass[10pt]{article}
% Include statements
\usepackage{graphicx}
\usepackage{amsfonts,amssymb,amsmath,amsthm}
\usepackage[sort]{natbib}
\usepackage[margin=1in,nohead]{geometry}
\usepackage{multirow,rotating,array}
\usepackage{algorithm,algorithmic}
\usepackage{pdfsync}
\usepackage{hyperref}
\usepackage{braket}
\usepackage{mathtools}
\usepackage{todonotes}
\hypersetup{backref,colorlinks=true,citecolor=blue,linkcolor=blue,urlcolor=blue}
\renewcommand{\qedsymbol}{$\blacksquare$}
\setlength{\parindent}{0cm}
\setlength{\parskip}{10pt}


% For sequential numbering
\newcounter{lecnum}
\renewcommand{\thepage}{\thelecnum\ -\ \arabic{page}}
\renewcommand{\thesection}{\thelecnum.\arabic{section}}
\renewcommand{\theequation}{\thelecnum.\arabic{equation}}
\renewcommand{\thefigure}{\thelecnum.\arabic{figure}}
\renewcommand{\thetable}{\thelecnum.\arabic{table}}
\newcommand{\eqdef}{=\vcentcolon}



% Theorem environments (\autoref compatible)
\usepackage{aliascnt}
\newtheorem{theorem}{Theorem}[lecnum]

\newaliascnt{result}{theorem}
\newtheorem{result}[theorem]{Result}
\aliascntresetthe{result}
\providecommand*{\resultautorefname}{Result}
\newaliascnt{lemma}{theorem}
\newtheorem{lemma}[lemma]{Lemma}
\aliascntresetthe{lemma}
\providecommand*{\lemmaautorefname}{Lemma}
\newaliascnt{prop}{theorem}
\newtheorem{proposition}[prop]{Proposition}
\aliascntresetthe{prop}
\providecommand*{\propautorefname}{Proposition}
\newaliascnt{cor}{theorem}
\newtheorem{corollary}[cor]{Corollary}
\aliascntresetthe{cor}
\providecommand*{\corautorefname}{Corollary}
\newaliascnt{conj}{theorem}
\newtheorem{conjecture}[conj]{Conjecture}
\aliascntresetthe{conj}
\providecommand*{\conjautorefname}{Corollary}
\newaliascnt{def}{theorem}
\newtheorem{definition}[def]{Definition}
\aliascntresetthe{def}
\providecommand*{\defautorefname}{Definition}
\newaliascnt{ex}{theorem}
\newtheorem{example}[ex]{Example}
\aliascntresetthe{ex}
\providecommand*{\exautorefname}{Example}


\newtheorem{assumption}{Assumption}
\renewcommand{\theassumption}{\Alph{assumption}}
\providecommand*{\assumptionautorefname}{Assumption}

\def\algorithmautorefname{Algorithm}
\renewcommand*{\figureautorefname}{Figure}%
\renewcommand*{\tableautorefname}{Table}%
\renewcommand*{\partautorefname}{Part}%
\renewcommand*{\chapterautorefname}{Chapter}%
\renewcommand*{\sectionautorefname}{Section}%
\renewcommand*{\subsectionautorefname}{Section}%
\renewcommand*{\subsubsectionautorefname}{Section}% 


% My Macros
\def\indep{\perp\!\!\!\perp}
\newcommand{\given}{\mbox{ }\vert\mbox{ }}
\newcommand{\F}{\mathcal{F}}
\newcommand{\Expect}[1]{\mathbb{E}\!\left[#1\right]}
\renewcommand{\P}{\mathbb{P}}
\newcommand{\R}{\mathbb{R}}
\newcommand{\X}{\mathcal{X}}
\newcommand{\B}{\mathcal{B}}
\DeclareMathOperator*{\Variance}{Var}
\newcommand{\Var}[1]{\Variance\!\left[#1\right]}
\DeclareMathOperator*{\Covariance}{Cov}
\newcommand{\Cov}[1]{\Covariance\!\left[#1\right]}
\newcommand{\Y}{\mathcal{Y}}
\newcommand{\norm}[1]{\left\lVert #1 \right\rVert}
\newcommand{\email}[1]{\href{mailto:#1}{#1}}
\DeclareMathOperator*{\argmin}{argmin}
\DeclareMathOperator*{\argmax}{argmax}
\newcommand{\indicator}{\mathbbm{1}}
\newcommand{\cdist}{\rightsquigarrow}
\newcommand{\cprob}{\xrightarrow{P}}
\newcommand{\clp}{\xrightarrow{L_p}}
\newcommand{\cas}{\xrightarrow{as}}
\renewcommand{\bar}{\overline}
\renewcommand{\hat}{\widehat}


% Your new macros



% To be entered
\setcounter{lecnum}{6}
\newcommand{\lecturer}{Prof.\ McDonald}
\newcommand{\scribe}{Nikolai Karpov}
\newcommand{\chtitle}{Duality}
\newcommand{\lecdate}{7 August 2017}


\begin{document}
\rule{6.5in}{1pt}

\textsc{STAT--S 782
        \hfill \thelecnum\ --- \chtitle
        \hfill \lecdate}

\textsc{Lecturer: \lecturer \hfill Scribe: \scribe}
\rule{6.5in}{1pt}

\begin{example}[Max-Flow and Min-Cut]
	For a graph $G = (V, E)$ with an edge limit function $c : E \rightarrow \R_{\geq 0}$ (sometimes this pair is called network) and vertices $v_s, v_t$; source and sink
	accordingly. The Max-Flow is the following problem.
	\[\left\{
	\begin{array}{lll}
	& \max\limits_{x \in \R^E} {\sum\limits_{j \in V}f_{sj}} \\
	\text{s.t.} &  0 \leq f_{ij} \leq c_{ij} & \forall (i, j) \in E \\
	& \sum\limits_{(i, k) \in E} f_{ik} = \sum\limits_{(k, i) \in E} f_{ki} & \forall k \in V \setminus \{v_s, v_t\}
	\end{array}
	\right.
	\]
	Dual version of this problem:
	\[
	\left\{
	\begin{array}{lll}
	&\min\limits_{(b, x) \in \R^E\times \R^V}\sum\limits_{(i, j) \in E}b_{ij}c_{ij} \\
	\text{s.t.} & b_{ij}+x_j-x_i \geq 0 & \forall (i, j) \in E \\
	& b \geq 0, x_s = 1, x_t = 0
	\end{array}
	\right.
	\]
\end{example}
	Suppose $x \in \{0, 1\}^V$ at the solution. If we consider two sets
	$A = \{i : x_i = 0\}$ and $B = \{i : x_i = 1\}$ then inequalities
	$b_{ij} \geq x_i - x_j$ is equivalent to $b_{ij} \geq I(i \in A, j \in B)$. Obviously, the optimum $b$ this inequalities are equalities $b_{ij} = I(i \in A, j \in B)$. We can reformulate our problem as find of partition (cut) $V$ into two disjoint sets $A$ and $B$ such that $v_s \in A$, $v_t \in B$ and $\sum\limits_{i \in A, j \in B}c_{ij} \rightarrow \min$; the last term is called \emph{capacity of the cut $(A, B)$}. Any such cut gives an upper bound on the value of max-flow in the network.
	
	Our Dual of Max-Flow was $LP$-relaxation of this problem. Such wise $$\text{value of Max-Flow} \leq \text{optimal of }LP\text{-relaxation} \leq \text{capacity of Min-Cut}\,.$$
	\begin{theorem}[Max-Flow Min-Cut Theorem] The optimal flow is equal to the capacity of an optimal cut.
	\end{theorem}
	In this case Primal and Dual gas the same optimal values. It's truth in general case for linear programs (see Farkas' lemma in~\cite{boyd2004convex}); this property is called ``strong duality''.

\section{Lagrangian}
\subsection{Linear programs}
Return to $LP$ (alternate derivation)
\[
\left\{
\begin{array}{lll}
&\min\limits_xc^Tx\\
\text{s.t.} & Ax = b \\
& Gx \geq h
\end{array}
\right.
\]

For any $u$ and $v \geq 0$ if $x$ is feasible:
\[c^Tx \geq c^Tx+u^T\left(Ax-b\right)+v^T\left(Gx-h\right) \eqdef L(x, u, v)\,.\]
If $C$ is a feasible set, $f^*$ is optimal value:
\[f^* \geq \min\limits_{x \in C}L(x, u, v)\geq\min\limits_{x}L(x, u, v)\eqdef g(u, v)\,.\]

\[
g(u, v) = \left\{
	\begin{array}{ll}
	-b^Tu-h^Tv &\text{, if } c = -A^Tu-G^Tv \\
	-\infty &\text{, otherwise}
	\end{array}
\right.
\]

Maximize $g$ over $u$, $v \geq 0$ provide lower bound.
We call $L(x, u, v)$ the Lagrangian, $g(u, v)$ the Lagrange dual function.

\subsection{General Case}

\[
\left\{
\begin{array}{lll}
&\min{f(x)} \\
\text{s.t.} & h_i(x) \leq 0 & i \in [m] \\
& l_j(x) = 0 & i \in [r]
\end{array}
\right.
\]
\[L(x, u, v) = f(x) + \sum\limits_{i=1}^mu_ih_i(x)+\sum\limits_{j=1}^rv_jl_j(x)~\left(u \geq 0\right)\] 

\[f^* \geq \min_{x}{L(x, u, v)} \eqdef g(u, v)\]

$g$ is concave.

\begin{example}
	\[
	\left\{
	\begin{array}{lll}
	&\min\limits_{x}\frac{1}{2}x^TQx+c^Tx \\
	\text{s.t.} & Ax=b \\
	& x \geq 0
	\end{array}
	\right.
	\]
	\[L(x, u, v) = \frac{1}{2}x^TQx+c^Tx-u^Tx+v^T(Ax-b)\]
	\[x^* = -Q^{-1}(c-u^TA^Tv)\]
	\[L(x^*,u,v) = -\frac{1}{2}(c-u+A^Tv)Q^{-1}(c-u+A^Tv)-b^Tv\]
	what if $Q \succeq 0\,?$
	\begin{itemize}
		\item \textbf{Case 1:} $c -u + A^Tv \in col(Q)$
		\[x = -Q^T(c-u+A^Tv) \implies P(c-u+A^Tv)=0\] where P is projection onto $nul(Q)\,.$
		\item \textbf{Case 2:} $c-u+A^Tv \not\in col(Q) \implies c-u+A^Tv = z_1 + z_2$ where $z_1 \in col(Q), z_2 \in nul(Q)$ and $z_2 \not= 0\,.$ 
	\end{itemize}
	\[
	g(u, v) = \left\{
	\begin{array}{ll}	
	-\frac{1}{2}(c-u+A^Tv)Q^{-1}(c-u+A^Tv)-b^Tv & \text{, Case 1} \\
	-\infty& \text{, Case 2}
	\end{array}
	\right.
	\]
\end{example}

\bibliographystyle{scribebibsty}
\bibliography{s782references}

\end{document}
