\documentclass[10pt]{article}
% Include statements
\usepackage{graphicx}
\usepackage{amsfonts,amssymb,amsmath,amsthm}
\usepackage[sort]{natbib}
\usepackage[margin=1in,nohead]{geometry}
\usepackage{multirow,rotating,array}
\usepackage{algorithm,algorithmic}
%\usepackage{pdfsync}
\usepackage{hyperref}
\hypersetup{backref,colorlinks=true,citecolor=blue,linkcolor=blue,urlcolor=blue}
\renewcommand{\qedsymbol}{$\blacksquare$}
\setlength{\parindent}{0cm}
\setlength{\parskip}{10pt}


% For sequential numbering
\newcounter{lecnum}
\renewcommand{\thepage}{\thelecnum\ -\ \arabic{page}}
\renewcommand{\thesection}{\thelecnum.\arabic{section}}
\renewcommand{\theequation}{\thelecnum.\arabic{equation}}
\renewcommand{\thefigure}{\thelecnum.\arabic{figure}}
\renewcommand{\thetable}{\thelecnum.\arabic{table}}



% Theorem environments (\autoref compatible)
\usepackage{aliascnt}
\newtheorem{theorem}{Theorem}[lecnum]

\newaliascnt{result}{theorem}
\newtheorem{result}[theorem]{Result}
\aliascntresetthe{result}
\providecommand*{\resultautorefname}{Result}
\newaliascnt{lemma}{theorem}
\newtheorem{lemma}[lemma]{Lemma}
\aliascntresetthe{lemma}
\providecommand*{\lemmaautorefname}{Lemma}
\newaliascnt{prop}{theorem}
\newtheorem{proposition}[prop]{Proposition}
\aliascntresetthe{prop}
\providecommand*{\propautorefname}{Proposition}
\newaliascnt{cor}{theorem}
\newtheorem{corollary}[cor]{Corollary}
\aliascntresetthe{cor}
\providecommand*{\corautorefname}{Corollary}
\newaliascnt{conj}{theorem}
\newtheorem{conjecture}[conj]{Conjecture}
\aliascntresetthe{conj}
\providecommand*{\conjautorefname}{Corollary}
\newaliascnt{def}{theorem}
\newtheorem{definition}[def]{Definition}
\aliascntresetthe{def}
\providecommand*{\defautorefname}{Definition}
\newaliascnt{ex}{theorem}
\newtheorem{example}[ex]{Example}
\aliascntresetthe{ex}
\providecommand*{\exautorefname}{Example}


\newtheorem{assumption}{Assumption}
\renewcommand{\theassumption}{\Alph{assumption}}
\providecommand*{\assumptionautorefname}{Assumption}

\def\algorithmautorefname{Algorithm}
\renewcommand*{\figureautorefname}{Figure}%
\renewcommand*{\tableautorefname}{Table}%
\renewcommand*{\partautorefname}{Part}%
\renewcommand*{\chapterautorefname}{Chapter}%
\renewcommand*{\sectionautorefname}{Section}%
\renewcommand*{\subsectionautorefname}{Section}%
\renewcommand*{\subsubsectionautorefname}{Section}% 


% My Macros
\def\indep{\perp\!\!\!\perp}
\newcommand{\given}{\mbox{ }\vert\mbox{ }}
\newcommand{\F}{\mathcal{F}}
\newcommand{\Expect}[1]{\mathbb{E}\!\left[#1\right]}
\renewcommand{\P}{\mathbb{P}}
\newcommand{\R}{\mathbb{R}}
\newcommand{\X}{\mathcal{X}}
\newcommand{\B}{\mathcal{B}}
\DeclareMathOperator*{\Variance}{Var}
\newcommand{\Var}[1]{\Variance\!\left[#1\right]}
\DeclareMathOperator*{\Covariance}{Cov}
\newcommand{\Cov}[1]{\Covariance\!\left[#1\right]}
\newcommand{\Y}{\mathcal{Y}}
\newcommand{\norm}[1]{\left\lVert #1 \right\rVert}
\newcommand{\email}[1]{\href{mailto:#1}{#1}}
\DeclareMathOperator*{\argmin}{argmin}
\DeclareMathOperator*{\argmax}{argmax}
\newcommand{\indicator}{\mathbbm{1}}
\newcommand{\cdist}{\rightsquigarrow}
\newcommand{\cprob}{\xrightarrow{P}}
\newcommand{\clp}{\xrightarrow{L_p}}
\newcommand{\cas}{\xrightarrow{as}}
\renewcommand{\bar}{\overline}
\renewcommand{\hat}{\widehat}


% Your new macros



% To be entered
\setcounter{lecnum}{6}
\newcommand{\lecturer}{Prof.\ McDonald}
\newcommand{\scribe}{Arash Khodadadi}
\newcommand{\chtitle}{KKT conditions}
\newcommand{\lecdate}{12 September 2017}


\begin{document}
\rule{6.5in}{1pt}

\textsc{STAT--S 782
        \hfill \thelecnum\ --- \chtitle
        \hfill \lecdate}

\textsc{Lecturer: \lecturer \hfill Scribe: \scribe}
\rule{6.5in}{1pt}

In what follows P and D stand for the primal and the dual problems, respectively.

\section{Weak duality}
We have:

\begin{equation}
\begin{aligned}
	f(x) & \ge  f(x)+u^Th(x)+v^Tl(x):=L(x,u,v)\\
		 & \ge  \min_x L(x,u,v):=g(u,v)
\end{aligned}
\end{equation}

\noindent and so:

\begin{equation}
	f^* \ge g(u,v) \Longrightarrow f^* \ge g^*(u,v)
\end{equation}

This is called the weak duality.

\textbf{Note:} The dual is always convex (even if P is not).

\textbf{proof:}

\begin{equation}
\begin{aligned}
 g(u,v) & =\min_x L(x,u,v)= \min_x \{ f(x)+u^Th(x)+v^Tl(x) \}\\
 		& =-\max_x \{ f(x)+u^Th(x)+v^Tl(x) \}
\end{aligned}
\end{equation}

This is a pointwise maximization of convex functions of $u,v$, and so it is convex in $u,v$.

\section{Strong duality}

\begin{equation}
f^*=g^*
\end{equation}

When does this hold?

\textbf{Slater's conditions:} If P is convex and there exists at least one strictly feasible $x$, i.e., $h_i(x)<0$, then we have strong duality.

\textbf{An important extension:} We only need this condition for the non-affine $h_i(x)$.

\subsection{Strong duality in LP}
In LP we have:
\begin{itemize}
	\item Dual of dual is P
	\item Strong duality if P is feasible
	\item Strong duality if D is feasible
	\item The previous two points imply that we have strong duality unless both P and D are infeasible.  
\end{itemize}
 
\begin{example}[SVM] 
	\begin{equation}
	\begin{aligned}
		\min_{\zeta,\beta,\beta_0} \frac{1}{2} ||\beta||_2^2+C\sum_i \zeta_i\\
		s.t \,\,\, \zeta_i \ge 0 \, , \, y_i(x_i^T \beta + \beta_0) \ge 1-\zeta_i 
	\end{aligned}
	\end{equation}
\end{example} 

The dual of this is:

\begin{equation}
\begin{aligned}
	\max_w -\frac{1}{2} w^T\tilde{X}^T\tilde{X}w+1^Tw\\
	s.t \,\,\, 0 \le w \le C1 \,\,\, ,\,\,\, w^Ty=0
\end{aligned}
\end{equation}

\noindent where $\tilde{X}=diag(y)X$.

Clearly, $w=0$ is dual feasible and so is primal feasible.

\section{KKT conditions}
1. Stationarity:  $0 \in \partial(f(x)+u^Th(x)+v^Tl(x))$

	For some pair $(u,v)$, $x$ minimizes the Lagrangian. 

2. Complementary slackness: $u_ih_i(x)=0 \,\,\, , \forall i$

3. Primal feasibility: $h_i(x) \le 0 \,\,,\,\, l_i(x)=0$

4. Dual feasibility:  $u \ge 0$


\begin{theorem}[Necessity]
If $x^*$ and $(u^*,v^*)$ are optimal and $f^*=g^*$, then they satisfy KKT conditions.
\end{theorem}

\textbf{Proof:} (i) 

\begin{equation}
\begin{aligned}
f(x^*)=g(u^*,v^*) & \le \min_x f(x)+u^Th(x)+v^Tl(x)\\
				  & \le f(x^*)+{u^*}^T h(x^*)+{v^*}^T l(x^*)\\
				  & \le f(x^*)
\end{aligned}
\end{equation} 

Now replace $\le$ with $=$. So from the second (in)equality we see that $X^*$ is the minimizer of the Lagrangian. Also, from the last (in)equality we see that ${u^*}^T h(x^*)=0$ and we have $u^* \ge 0$ and so we get the complementary slackness.

\begin{theorem}[Sufficiency]
	If $x^*$ and $(u^*,v^*)$ satisfy KKT conditions then they are P and D optimal and $f^*=g^*$.
\end{theorem}

\begin{example}[SVM]
	KKT conditions:
	\begin{enumerate}
		\item Stationarity: $w^Ty=0 \,\,,\,\, \beta=w^T\tilde{X} \,\,,\,\, w=C1-v$
		\item CS: $v_i\zeta_i=0 \,\,,\,\, w_i(1-\zeta_i-y_i(x_i^T\beta+\beta_0))=0$
	\end{enumerate}
\end{example}

\begin{example}[constrained and Lagrangian forms]
	When are the two following forms equivalent?
	
	constrained form (C):
	\begin{equation}
	\begin{aligned}
		\min f(x)\\
		s.t. \,\, h(x) \le t
	\end{aligned}
	\end{equation}
	
	Lagrangian form (L):
	\begin{equation}
	\begin{aligned}
	\min f(x)+\lambda h(x)
	\end{aligned}
	\end{equation}

When C is strictly feasible, strong duality holds. So there exists $\lambda$ such that for each $x$ that solves C those $x$ minimize L.

Now, if $x^*$ solves L, then KKT condition for C hold by taking $t=h(x^*)$ and so $x^*$ is a solution of C.
\end{example}

   
\bibliographystyle{scribebibsty}
\bibliography{s782references}

\end{document}
